\documentclass{article}

\input BylawsPreamble

\title{Bylaws of Sigma Zeta of ΛΧΑ}
\date{November 17, 2018 draft}

\begin{document}

\maketitle

\section*{Editorial Note}

These bylaws refer to Lambda Chi Alpha’s \emph{Constitution} and \emph{Statutory
Code} in explanatory margin notes \marginnote{This is a margin note.} (example
at right). Unless otherwise noted, references to the \emph{Constitution} and
\emph{Statutory Code} are to the 44th~(2018) edition, available at
\url{https://www.lambdachi.org/aboutlca-2/policies/}.

% The U-M Center for Campus Involvement requires that student organization
% governing documents contain the organization’s name and its “Mission and
% purpose”; see
% https://campusinvolvement.umich.edu/content/student-organization-registration-0#Constitution%20Requirements.
\section{Name}

The official name of this Chapter shall be: Sigma Zeta \marginnote{In Lambda Chi
Alpha, \emph{Zeta} has roughly the same meaning as \emph{chapter}. Lambda Chi
Alpha’s board of directors assigns names to Zetas according to
\ConstitutionReference{3}{5}.} of Lambda Chi Alpha Fraternity at the University
of Michigan in Ann Arbor, MI.

\section{Objects}

The purposes of this Chapter shall be: To maintain at the University of Michigan
in Ann Arbor, Michigan, a Chapter of the Lambda Chi Alpha Fraternity, in
accordance with its ideals, ritual, standards, traditions, and laws.

\section{Chapter Governance}

This Chapter shall be governed by the following laws:

\hspace*{-\tabcolsep}%
\begin{tabular}{ll}
  First:  & \emph{Constitution} and \emph{Statutory Code} \\
  Second: & Ritual                                        \\
  Third:  & Orders of the Grand High Zeta                 \\
  Fourth: & University of Michigan                        \\
  Fifth:  & NIC/IFC Bylaws                                \\
  Sixth:  & Chapter Bylaws                                \\
\end{tabular}

\section{Organization}

\subsection{Executive Committee}

The Executive Committee shall have the powers and duties of the Active Chapters
as defined in \StatutoryCodeReference{3}{9}.

\begin{subsubsectionList}
  \item A pool of candidates shall be nominated and seconded for each member at
  large and alternate position popular vote following an election shall
  determine the winner of each slot.

  \item Nomination for one Executive Committee position shall not preclude the
  nomination to any other position, Executive or otherwise.
\end{subsubsectionList}

\subsection{Chapter Committees}

Chapter committees shall exist to promote and encourage involvement of the
general membership and to carry out the programs and operations of Sigma Zeta.
These committees shall be:

Executive Committee\\
Risk Management Task Force \marginnote{The Risk Management Task Force is
described in \StatutoryCodeReference{III}{20} of the 41st (2012) edition of the
\emph{Statutory Code}.}\\
Committees appointed at the prerogative of an officer of the High Zeta, provided
that such committees follow all codes of the \emph{Constitution} and
\emph{Statutory Code} and are approved by the Executive Committee

\section{Chapter Meetings}

\subsection{Chapter Business Meetings}

Chapter business meetings shall be held weekly during the school year, with the
day and time to be determined by the Chapter.

\subsection{Formal Business Meetings}

Formal business meetings will be held at the first meeting of every month during
the academic year. Dress code for formal business meetings shall consist of
collared shirt, tie, slacks and dress shoes unless deemed otherwise by the High
Alpha.

% Code 3‑12 says that “[a] majority of the Collegiate Brothers and Associate
% Members of a Chapter shall constitute a quorum, unless a greater number is
% provided for in the Chapter bylaws.”
\subsection{Quorum}

A majority of members in good standing shall constitute a quorum \marginnote{
\StatutoryCodeReference{3}{12} describes the quorum of a chapter meeting.}
(membership greater than fifty percent). Action may be taken at any meeting by a
simple majority of those present and voting, unless otherwise specified by the
\emph{Constitution} and \emph{Statutory Code}, these Chapter Bylaws, or
\emph{Robert’s Rules of Order}.

\subsection{Attendance}

Attendance at Chapter meetings is required. Policies and procedures governing
attendance at Chapter meetings shall be as prescribed by
Article~\ref{Attendance Policy} of these Bylaws. Visiting and alumni brothers
may attend chapter meetings but may not vote. Guests wishing to address the
Chapter may do so only after permission is obtained from the High Alpha or the
Active Chapter.

\subsection{Decorum}

Chapter meetings shall be structured using the following procedures:

\begin{subsubsectionList}
  \item {\titleStyle Agenda\titleSuffix}
  All High Zeta officers must submit their reports for the agenda to the High
  Gamma 24~hours before the start of chapter, after which the High Gamma will
  compile an agenda. Failure to submit an agenda for 2~consecutive weeks will
  result in a mandatory meeting with the Executive Committee.

  \item {\titleStyle Enforcing Decorum\titleSuffix}
  The High Alpha is responsible for maintaining decorum at Chapter meetings and
  may designate a Sergeant-at-Arms to enforce decorum. Actions which violate
  decorum at Chapter meetings include eating, smoking or chewing tobacco, use of
  profanity, or disorderly conduct. The High Alpha shall have the discretionary
  authority to dismiss a member who, in his judgment, is in violation of proper
  decorum.
\end{subsubsectionList}

\section{Chapter Administration}

\subsection{Chapter Officers}

The following officers shall be elected to administer the affairs of the
Chapter. In order of rank, they are:

\hspace{-\tabcolsep}%
\begin{tabular}{ll}
  High Alpha   & President             \\
  High Beta    & First Vice President  \\
  High Theta   & Second Vice President \\
  High Gamma   & Secretary             \\
  High Tau     & Treasurer             \\
  High Iota    & Risk Manager          \\
  High Rho     & Alumni Liaison        \\
  High Kappa   & Fraternity Educator   \\
  High Delta   & Recruitment Chairman  \\
  High Phi     & Ritualist             \\
  High Sigma   & Educational Chairman  \\
  High Epsilon & Social Chairman       \\
\end{tabular}

The High Zeta shall include these additional officers elected by the Chapter, in
order of rank after the lowest-ranked High Zeta officer prescribed by the
\emph{Statutory Code}:

House Manager\\
Intramural Sports Chairman\\
Greek Week Representatives

\subsection{Election of Officers}

Chapter elections shall be held annually on the second Sunday of November under
the supervision of the High Alpha or his designated representative. The
elections may be split into two halves, with the second half occurring on the
third Sunday of November.

\begin{subsubsectionList}
  \item {\titleStyle Nominations\titleSuffix}
  Nominations shall be on the first Sunday of November and shall lie on the
  table for at least one week. Additional nominations may be made at the time of
  elections.

  \item {\titleStyle Eligibility\titleSuffix}
  To be eligible to hold an elective or appointive office, a member must be in
  good standing, academically and financially, as defined by the
  \emph{Constitution} and \emph{Statutory Code} (IV‑1, III‑3) and these Bylaws.

  \item {\titleStyle Election Order\titleSuffix}
  Officers shall be elected in order of rank.

  \item {\titleStyle Speeches\titleSuffix}
  All candidates will speak in alphabetical order. The speeches for High Alpha
  candidates will not exceed five minutes in duration. The speeches for all
  other officer candidates will not exceed three minutes in duration. During the
  election of each office, the candidates will wait outside the Chapter room,
  entering only when called to speak or to vote.

  \item {\titleStyle Voting\titleSuffix}
  The vote will be by secret, written ballot.

  \item {\titleStyle Votes Required for Election\titleSuffix}
  An individual may not be elected to Chapter office unless he has received a
  majority vote of those present and voting. Should there fail to be a majority
  for a candidate on any ballot, the High Alpha shall call for a second ballot,
  from which the name of the candidate polling the lowest vote on the first
  ballot shall be dropped. Should the second ballot likewise fail to produce a
  majority for any candidate, subsequent ballots shall be called for until such
  a majority is secured and on each subsequent ballot the candidate polling the
  lowest vote on the preceding ballot shall be dropped. The High Alpha shall
  vote only to break a tie on any ballot.

  \item {\titleStyle Tallying of Votes\titleSuffix}
  The High Alpha and the High Phi shall tally votes. Should the High Alpha be
  absent or should his name appear on the ballot under consideration, the
  officer of highest rank in attendance or whose name does not appear on the
  ballot under consideration shall tally votes in addition to the High Phi.
  Should the High Phi be absent or should his name appear on the ballot under
  consideration, the officer of highest rank in attendance or whose name does
  not appear on the ballot under consideration shall tally votes in addition to
  the High Alpha. Should both the High Alpha and the High Phi be absent or
  should both of their names appear on the ballot under consideration, the
  officers of highest rank in attendance or whose names do not appear on the
  ballot under consideration shall tally votes.

  \item {\titleStyle Resignation from Office\titleSuffix}
  To resign his position, an elective or appointive officer must appear at a
  Chapter meeting and read aloud a written Statement of Resignation. If his
  appearance is not possible, a letter of resignation, addressed to the Chapter,
  shall suffice.

  \item {\titleStyle Vacancies\titleSuffix}
  In the event of a vacancy, the position may be filled by Executive Committee
  appointment, or upon an objection and a second by the Chapter, a special
  election may be called.
\end{subsubsectionList}

\subsection{High Pi}

The High Pi shall serve as Chancellor of the Active Chapter, Advisor to the
Chapter and Chief Judicial Officer. He shall serve a two-year term and shall be
appointed by the Grand High Pi upon recommendation of the Active Chapter of
Sigma Zeta.

\subsection{Executive Committee}

The Executive Committee shall be composed of the High Alpha, High Tau, High Pi,
and two active members elected by the Chapter. One alternate member shall be
elected and shall attend all Executive Committee meetings and exercise voting
privileges in the absence of any of the five standing Executive Committee
members. Unless elected as a regular or alternate member, the High Beta, the
High Theta and High Sigma shall serve as ex-officio members and shall not have a
vote unless they have been elected as active or alternate members. In the event
that one of these ex-officio members is elected as an active or alternate member
with a vote, another member will be elected to the executive committee as an
ex-officio member, thus maintaining the number of members on exec to 8 (5 voting
and 3 ex-officio) plus one alternate. The High Beta shall exercise full voting
privileges in the absence of the High Alpha.

\subsection{Educational Advisor}

The Chapter shall elect an Educational Advisor who shall be a member of the
University of Michigan faculty whenever possible. He or she need not be an
alumnus of Lambda Chi Alpha.

\subsection{Special Events}

Prior to any large event (i.e. social or philanthropy) at the house, there must
be a Risk Management Task Force meeting to discuss the plans and procedures for
set-up, during, and teardown for the event. The officer holding the event must
meet with the Executive Committee at their next meeting after the event to
discuss the success and/or failure of the event.

\section{Membership}

\subsection{Status}

Membership in the Active Chapter shall include active members (initiated) and
associate members (non-initiated) who are students at the University of
Michigan. To actively participate in chapter affairs, a member (active or
associate) must be in good standing in accordance with Codes IV‑1 (Good
Standing), VIII‑7 (Financial), and VIII‑8 (Scholastic) of the \emph{Statutory
Code}.

\subsection{Chapter Obligations}

As required by the General Fraternity, the Chapter shall maintain the following
Operating Standards as enumerated in Article~III, Sec.~3 of the
\emph{Constitution}:

\begin{subsubsectionList}
  \item The Chapter Membership, active and associate, must be equal in number to
  the campus fraternity average or 40, whichever is smaller, but in no event
  fewer than 20.

  \item The Chapter must maintain an annual cumulative 2.5 Grade Point Average
  or a \acronym{GPA} above the All-Men’s average, whichever is lesser.

  \item At least 80\% of the members must be engaged in extracurricular
  activities.

  \item The chapter must be financially solvent, have an adequate accounting
  system and operate under an approved budget.

  \item Each member shall pay the equivalent of at least \$4.00 per month to a
  Reserve Fund, which shall be held and controlled by the House Corporation, to
  be used for housing needs.

  \item The Chapter must have a complete set of ritual equipment.

  \item The Chapter must have a High Pi and a functioning alumni organization of
  at least three members.

  \item The chapter must have a set of bylaws, updated every two years and
  approved by the General Fraternity.

  \item The Chapter must have at least one High Zeta member and one other member
  at every General Assembly and Leadership Seminar, and at least one High Zeta
  Member and two other members at every regional conference.

  \item The Chapter must have its house inspected annually by a professionally
  certified Fire Marshal.

  \item The Chapter must implement a Standards for Chapter Excellence Program,
  including the appointment or election of a Standards Chairman.
\end{subsubsectionList}

\subsection{Individual Obligations}

Each member shall have the following responsibilities and obligations.

\begin{subsubsectionList}
  \item To abide by the laws, rulings, and accepted traditions of Sigma Zeta and
  Lambda Chi Alpha Fraternity.

  \item To pay promptly, all legally assessed dues, charges, and fines. (Code
  VIII‑7)

  \item To attend all regular meetings and all official functions of the Chapter
  whenever possible.

  \item To put forth a definite effort in all areas requiring member support.

  \item To be in good scholastic standing with the Fraternity. (Code VIII‑8)

  \item Each Associate Member shall complete sober monitor training or the
  equivalent, which is endorsed by the Interfraternity Council for its members,
  within their first semester of Association with Sigma Zeta. A failure to do so
  by their second semester as a Brother, if initiated, or as an Associate, if
  still associated, will result in a prohibition from all social events, Epsilon
  events, or the like as determined by the Executive Committee, until such
  training has been completed.

  \item If a member of Lambda Chi Alpha obtains below the minimum \acronym{GPA}
  requirement in the last term completed then they will be placed on academic
  probation. In addition to the rules set forth in the \emph{Constitution} \&
  \emph{Statutory Code}, any member on academic probation is immediately
  restricted from social events held by the fraternity once the next term begins
  until the member has shown sufficient academic progress. Academic progress can
  be in the form of any grade received for an assignment or exam. The High Sigma
  and Executive Committee will determine whether the academic progress is
  considered sufficient enough to remove the member from social probation. The
  Executive Committee and The High Sigma have the right to place anyone back on
  social probation.
\end{subsubsectionList}

\subsection{Voting On Candidates for Membership}

\begin{subsubsectionList}
  \item {\titleStyle Association\titleSuffix}
  A 2/3 favorable vote of those active and associate members present and voting
  shall be required for a prospective member to be associated into the Chapter.
  Voting procedures are outlined in Code IV‑3 of the \emph{Statutory Code}.
  Scholastic requirements are outlined in Code IV‑4 of the \emph{Statutory
  Code}. The High Kappa shall inform Associate Members of his subjective opinion
  of their progress within the association process three weeks prior to the
  initiation vote. A bid will be granted based on one of two methods:

  \begin{orderedList}
    \item Formal bid meeting.

    \item If the High Delta so chooses, he may divide the chapter into
    recruitment groups of four or more. With the unanimous vote of the group and
    the approval of the High Delta, a bid will be given. This process is not
    applicable if the potential new member voted on in a formal bid meeting.
  \end{orderedList}

  \item {\titleStyle Initiation\titleSuffix}
  A 75\% favorable vote of those active members present and voting shall be
  required for an associate member to be initiated into the Chapter. Voting
  procedures are outlined in Code IV‑3 of the \emph{Statutory Code}. Scholastic
  requirements are outlined in Code IV‑4 of the \emph{Statutory Code}.

  \item {\titleStyle Termination of Associate Membership\titleSuffix}
  Associate membership may be terminated at any time by the active members of
  the Chapter. A 75\% favorable vote of the members present and voting shall be
  necessary to continue associate membership.

  \item {\titleStyle Termination of Active Membership.}
  \begin{orderedList}
    \item {\titleStyle Life Membership}. Membership in Lambda Chi Alpha is for
    life and shall be terminated only by death, resignation, or expulsion.

    \item {\titleStyle Inactive Status}. Inactive status is permitted for an
    active brother who has paid dues and other charges for each of 8 full
    semesters, in accordance with Article~IV, Section~2 of the
    \emph{Constitution}.

    \item {\titleStyle Resignation}. A member who resigns his membership is
    considered automatically expelled. In order to resign his membership from
    the active chapter, a member must submit a written Statement of Resignation
    to the High Alpha.

    \item {\titleStyle Limitations}. A member on a supervision or probation
    status shall be subject to the limitations set forth by the Active Chapter
    and the Executive Committee, in accordance with Article~VIII of the
    \emph{Constitution} and \emph{Statutory Code}.
  \end{orderedList}
\end{subsubsectionList}

\subsection{Initiation}

\begin{subsubsectionList}
  \item {\titleStyle Time\titleSuffix}
  The exact time and place for the Initiation Ritual shall be designated by the
  High Phi.

  \item {\titleStyle Manner of Initiation\titleSuffix}
  This Chapter shall associate, initiate, and educate men only in strict
  accordance with the Laws of the Fraternity.

  \item {\titleStyle Compulsory Attendance\titleSuffix}
  Every active member shall be required to attend the entire Initiation Ritual.

  \item {\titleStyle Initiation Fee\titleSuffix}
  The Initiation Fee shall be paid 14 calendar days in advance of the initiation
  date.

  \item {\titleStyle Zeta Numbers\titleSuffix}
  Zeta numbers shall be assigned in alphabetical order within an initiation
  group, unless otherwise determined by the Active Chapter.
\end{subsubsectionList}

\section{Housing}

\subsection{House Rules}

Each year, the Executive Committee shall adopt a set of house rules, which
include residency requirements, cleaning obligations, parking rights, and other
material necessary to maintain the chapter house.

\subsection{House Residency}

In accordance with Article~IV, Sec.~6(d) of the \emph{Constitution} of Lambda
Chi Alpha Fraternity, every member in good standing is required to live in the
chapter house unless excused by the Executive Committee. Any member not residing
in the chapter house and who provides no proof of a valid excuse accepts any
expulsion, suspension, or other lawful penalty levied upon them by the Executive
Committee of Sigma Zeta of Lambda Chi Alpha.

The validity of excuses shall be determined by the Executive Committee and must
adhere to the guidelines enumerated here. Examples of valid excuses are
scholarships or affirmation as an Resident Advisor in which the member must
reside in University housing; the member’s parents will not allow the member to
reside within the chapter house (only valid after the Executive Committee has
spoken with the member’s parents); and other excuses concurrent with the
previous examples.

\subsection{Parlor Fees}

Parlor fees will be decided along with the chapter budget, at the discretion of
the Executive Committee.

\subsection{House Duties}

The House Manager upon his appointment to office will draft a document, which is
written and posted, stating the policies, duties and penalties pertaining to
house cleaning to be approved by majority vote at the second chapter meeting.

\subsection{Parking}

Procedures for allocating parking spots at the chapter house shall be covered in
the Chapter House Rules. Use of the chapter house parking lot is for residents
of the chapter house only, although alumni of Lambda Chi Alpha are exempt from
fines and other penalties accrued through parking at the chapter house. The High
Rho is responsible for informing alumni when other parking areas may be used.

\subsection{Summer Housing}

The Chapter shall designate a Summer President and Summer House Manager who
shall be in charge of the chapter house if the High Alpha and House Manager are
not in summer residence. All chapter laws, policies and procedures shall be in
full effect for the summer residents, whether or not they are members of Lambda
Chi Alpha. Weekly house cleaning duties and maintenance of the lawn and grounds
shall be the responsibility of all summer house residents, with appropriate
fines for non-performance of assigned duties. A meeting of summer residents
shall be held at the start of each summer term to advise everyone of these house
policies, with the High Pi or an alumni officer present, if possible.

\section{Finances}

\subsection{Budget}

Sigma Zeta shall operate under an established budget for Fall and Winter
semesters, using the official University Academic Calendars to determine the
semester time frames. Each officer shall be required to submit a departmental
budget for consideration of the Executive Committee. The final budget shall be
prepared by the Executive Committee and approved by the Active Chapter and House
Corporation.

\subsection{Fees and Dues}

\begin{subsubsectionList}
  \item {\titleStyle Chapter Dues.}
  Chapter dues shall be set each semester. Special discount or payment
  incentives may be provided as determined by the Executive Committee.

  \item {\titleStyle Associate Member Fee.}
  The General Fraternity Associate Member Fee shall be paid within one week of
  association.

  \item {\titleStyle Initiation Fee.}
  The General Fraternity Initiation Fee shall be paid by the associate member
  before the Initiation Ritual.
\end{subsubsectionList}

\subsection{Financial Hardship}

A member may be granted inactive membership status, in case of undue financial
hardship upon written petition approved by the Executive Committee, as provided
in Article~IV, Sec.~2(a) of the \emph{Constitution}.

\section{Discipline}

A member has the right to bring before the Executive Committee a complaint
against another member or a situation which warrants disciplinary action. The
Executive Committee will follow the procedures as outlined in Article~VIII of
the \emph{Constitution} and \emph{Statutory Code}.

\section{Attendance Policy}\label{Attendance Policy}

\subsection{Required Events}

All events and activities of Sigma Zeta are considered mandatory with the
exception of social events and those so deemed by the High Alpha. Unexcused
absences shall be dealt with by the Executive Committee.

\subsection{Initiation Attendance}

Any active Brother in good standing of Sigma Zeta of Lambda Chi Alpha who is not
present for the entire Initiation Ritual will be assessed a fine of \$500.00.
Only express written consent from the High Phi at least three~(3) weeks prior to
the Ritual will be considered an excuse, except for family emergencies.

\section{Amendments}

\subsection{Basis}

The process of revision of these Bylaws shall follow \emph{Robert’s Rules of
Order}.

\subsection{Revisions}

\begin{subsubsectionList}
  \item Revisions to these Bylaws can be introduced at any regularly scheduled
  Chapter business meeting.

  \item Revisions must lie on the table for one week before action is taken.

  \item A quorum must be present during voting.

  \item Revisions shall require a two-thirds majority to pass.
\end{subsubsectionList}

\subsection{Biennial Review}

Review of these Bylaws shall occur in the Fall following each Lambda Chi Alpha
General Assembly. The Parliamentarian will be charged with the responsibility of
overseeing the review.

\subsection{Equal Opportunity}

Lambda Chi Alpha is committed to a policy of equal opportunity for all persons
and does not discriminate on the basis of race, color, national origin, age,
marital status, sex, sexual orientation, gender identity, gender expression,
disability, religion, height, weight, or veteran status in its membership or
activities unless permitted by university policy for gender specific
organizations. Upon joining the organization, all members agree not to undermine
the purpose or mission of Lambda Chi Alpha. Lambda Chi Alpha meets the
requirements for a Title~IX exemption.

\subsection{Constitution and Bylaw Re-ratification}

Lambda Chi Alpha will re-ratify its constitution at minimum once per three
calendar years.

\subsection{Sexual Misconduct Rules and Prevention}

As a brotherhood, we assert that we will not tolerate any acts of sexual abuse,
sexual harassment, rape of any person, or allow for a culture promoting sexual
misconduct among our members. If a member is found to be guilty under the sexual
misconduct policy of the University of Michigan of any of these acts, they will
be immediately expelled. Per national policy, proactive prevention of these acts
will be done by holding a sexual misconduct presentation once a semester with
all members prior to the first social event following the recruitment period and
another risk management presentation throughout the presentation. If alleged
sexual misconduct occurs, another sexual misconduct prevention presentation will
occur. Presentations for sexual misconduct and risk management will be conducted
using University of Michigan resources such as \acronym{SAPAC} and
\acronym{CAPS}. In addition to this chapter policy, we will follow all
University of Michigan and national Lambda Chi Alpha policies regarding sexual
misconduct. If a member is accused of alleged sexual misconduct, we will follow
the judiciary procedure set by our nationals in Article~7 Sections 4–6 of our
national constitution for internal judiciary procedures and we will follow the
University of Michigan’s sexual misconduct policy as well.

\end{document}
